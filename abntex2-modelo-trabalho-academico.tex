%% abtex2-modelo-trabalho-academico.tex, v-1.9.7 laurocesar
%% Copyright 2012-2018 by abnTeX2 group at http://www.abntex.net.br/ 
%%
%% This work may be distributed and/or modified under the
%% conditions of the LaTeX Project Public License, either version 1.3
%% of this license or (at your option) any later version.
%% The latest version of this license is in
%%   http://www.latex-project.org/lppl.txt
%% and version 1.3 or later is part of all distributions of LaTeX
%% version 2005/12/01 or later.
%%
%% This work has the LPPL maintenance status `maintained'.
%% 
%% The Current Maintainer of this work is the abnTeX2 team, led
%% by Lauro César Araujo. Further information are available on 
%% http://www.abntex.net.br/
%%
%% This work consists of the files abntex2-modelo-trabalho-academico.tex,
%% abntex2-modelo-include-comandos and abntex2-modelo-references.bib
%%

% ------------------------------------------------------------------------
% ------------------------------------------------------------------------
% abnTeX2: Modelo de Trabalho Academico (tese de doutorado, dissertacao de
% mestrado e trabalhos monograficos em geral) em conformidade com 
% ABNT NBR 14724:2011: Informacao e documentacao - Trabalhos academicos -
% Apresentacao
% ------------------------------------------------------------------------
% ------------------------------------------------------------------------

\documentclass[
	% -- opções da classe memoir --
	12pt,				% tamanho da fonte
	openright,			% capítulos começam em pág ímpar (insere página vazia caso preciso)
	twoside,			% para impressão em recto e verso. Oposto a oneside
	a4paper,			% tamanho do papel. 
	% -- opções da classe abntex2 --
	%chapter=TITLE,		% títulos de capítulos convertidos em letras maiúsculas
	%section=TITLE,		% títulos de seções convertidos em letras maiúsculas
	%subsection=TITLE,	% títulos de subseções convertidos em letras maiúsculas
	%subsubsection=TITLE,% títulos de subsubseções convertidos em letras maiúsculas
	% -- opções do pacote babel --
	english,			% idioma adicional para hifenização
	french,				% idioma adicional para hifenização
	spanish,			% idioma adicional para hifenização
	brazil				% o último idioma é o principal do documento
	]{abntex2}


\usepackage{lmodern}			% Usa a fonte Latin Modern			
\usepackage[T1]{fontenc}		% Selecao de codigos de fonte.
\usepackage[utf8]{inputenc}		% Codificacao do documento (conversão automática dos acentos)
\usepackage{indentfirst}		% Indenta o primeiro parágrafo de cada seção.
\usepackage{color}				% Controle das cores
\usepackage{graphicx}			% Inclusão de gráficos
\usepackage{microtype} 			% para melhorias de justificação
\usepackage{lipsum}				% para geração de dummy text
\usepackage[brazilian,hyperpageref]{backref}	 % Paginas com as citações na bibl
\usepackage[alf]{abntex2cite}	% Citações padrão ABNT


% Configurações do pacote backref
% Usado sem a opção hyperpageref de backref
\renewcommand{\backrefpagesname}{Citado na(s) página(s):~}
% Texto padrão antes do número das páginas
\renewcommand{\backref}{}
% Define os textos da citação
\renewcommand*{\backrefalt}[4]{
	\ifcase #1 %
		Nenhuma citação no texto.%
	\or
		Citado na página #2.%
	\else
		Citado #1 vezes nas páginas #2.%
	\fi}%


%-------------------------------------------------------------------------------------
%		DADOS PESSOAIS
%		Informações básicas sobre os trabalho e os autores 
%-------------------------------------------------------------------------------------
\titulo{Modelo Canônico de\\ Trabalho Acadêmico com \abnTeX}
\autor{Equipe \abnTeX}
\local{Salvador}
\data{2018, v-1.9.7}
\orientador{Lauro César Araujo}
\coorientador{Equipe \abnTeX}
\instituicao{%
  Universidade do Brasil -- UBr
  \par
  Faculdade de Arquitetura da Informação
  \par
  Programa de Pós-Graduação}
\tipotrabalho{Tese (Doutorado)} 
\preambulo{Modelo canônico de trabalho monográfico acadêmico em conformidade com
as normas ABNT apresentado à comunidade de usuários \LaTeX.}



%-------------------------------------------------------------------------------------
% 		CONFIGURAÇÃO PDF
%		Configurações de aparência do PDF final
%-------------------------------------------------------------------------------------

% alterando o aspecto da cor azul
\definecolor{blue}{RGB}{41,5,195}

% informações do PDF
\makeatletter
\hypersetup{
     	%pagebackref=true,
		pdftitle={\@title}, 
		pdfauthor={\@author},
    	pdfsubject={\imprimirpreambulo},
	    pdfcreator={LaTeX with abnTeX2},
		pdfkeywords={abnt}{latex}{abntex}{abntex2}{trabalho acadêmico}, 
		colorlinks=true,       		% false: boxed links; true: colored links
    	linkcolor=blue,          	% color of internal links
    	citecolor=blue,        		% color of links to bibliography
    	filecolor=magenta,      		% color of file links
		urlcolor=blue,
		bookmarksdepth=4
}
\makeatother


% Posiciona figuras e tabelas no topo da página quando adicionadas sozinhas
% em um página em branco. Ver https://github.com/abntex/abntex2/issues/170
\makeatletter
\setlength{\@fptop}{5pt} % Set distance from top of page to first float
\makeatother

% Possibilita criação de Quadros e Lista de quadros.
% Ver https://github.com/abntex/abntex2/issues/176
\newcommand{\quadroname}{Quadro}
\newcommand{\listofquadrosname}{Lista de quadros}

\newfloat[chapter]{quadro}{loq}{\quadroname}
\newlistof{listofquadros}{loq}{\listofquadrosname}
\newlistentry{quadro}{loq}{0}

% configurações para atender às regras da ABNT
\setfloatadjustment{quadro}{\centering}
\counterwithout{quadro}{chapter}
\renewcommand{\cftquadroname}{\quadroname\space} 
\renewcommand*{\cftquadroaftersnum}{\hfill--\hfill}

\setfloatlocations{quadro}{hbtp}

% O tamanho do parágrafo é dado por:
\setlength{\parindent}{1.3cm}

% Controle do espaçamento entre um parágrafo e outro:
\setlength{\parskip}{0.2cm}  % tente também \onelineskip


\makeindex

%%%%%%%%%%%%%%%%%%%%%%%%%%%%%%%%%%%%%%%%%%%%%%%%%%%%%%%%%%%%%%%%%%%%%%%%%%%%%%%%%%%%%%
%																					 %
% 							 Início do documento								     %
%																					 %
%%%%%%%%%%%%%%%%%%%%%%%%%%%%%%%%%%%%%%%%%%%%%%%%%%%%%%%%%%%%%%%%%%%%%%%%%%%%%%%%%%%%%%
\begin{document}

%\selectlanguage{english}
\selectlanguage{brazil}

% Retira espaço extra obsoleto entre as frases.
\frenchspacing 

%-------------------------------------------------------------------------------------
% 		ELEMENTOS PRÉ-TEXTUAIS
%		Insire elementos pré textuais, ficha catalográfica 
%-------------------------------------------------------------------------------------
% \pretextual

\imprimircapa
\imprimirfolhaderosto*	% (o * indica que haverá a ficha bibliográfica)


\include{pre-textuais/ficha_catalografica}
\include{pre-textuais/errata}
\include{pre-textuais/folha_aprovacao}
\include{pre-textuais/dedicatoria}
\include{pre-textuais/agradecimentos}
\include{pre-textuais/epigrafe}
\include{pre-textuais/resumo}
\include{pre-textuais/abstract}
\include{pre-textuais/listas_1}
\include{pre-textuais/listas_siglas}
\include{pre-textuais/listas_simbolos}

% ---
% inserir o sumario
% ---
\pdfbookmark[0]{\contentsname}{toc}
\tableofcontents*
\cleardoublepage
% ---



%-------------------------------------------------------------------------------------
% 		ELEMENTOS TEXTUAIS
%		Informações relevantes ao trabalho, capítulos e suas respectivas seções.
%-------------------------------------------------------------------------------------
\textual

\chapter{Introdução}

\section{Estado da Arte}

\section{Problema}

\section{Justificativa}

\section{Objetivos}

\subsection{Objetivo Geral}

\subsection{Objetivos Específicos}


\section{Organização}





\part{Preparação da pesquisa}

\chapter{Teoria}

\section{Quadros}



% \begin{quadro}[htb]
% \caption{\label{quadro_exemplo}Exemplo de quadro}
% \begin{tabular}{|c|c|c|c|}
% 	\hline
% 	\textbf{Pessoa} & \textbf{Idade} & \textbf{Peso} & \textbf{Altura} \\ \hline
% 	Marcos & 26    & 68   & 178    \\ \hline
% 	Ivone  & 22    & 57   & 162    \\ \hline
% 	...    & ...   & ...  & ...    \\ \hline
% 	Sueli  & 40    & 65   & 153    \\ \hline
% \end{tabular}
% \fonte{Autor.}
% \end{quadro}

% Primeira opção, utilizando \texttt{autoref}: Ver o \autoref{quadro_exemplo}. 
% Segunda opção, utilizando  \texttt{ref}: Ver o Quadro \ref{quadro_exemplo}.



\part{Referenciais teóricos}
% ---
% Capitulo de revisão de literatura
% ---
\chapter{Pesquisa}
% ---

% ---
\section{Aliquam vestibulum fringilla lorem}
% ---

\lipsum[1]


\part{Resultados}
\chapter{Resultados alcansados}

\section{Vestibulum}
% ---

\chapter{estudos funturos}



\section{Pellentesque sit amet pede ac sem eleifend consectetuer}




% ----------------------------------------------------------
% Finaliza a parte no bookmark do PDF
% para que se inicie o bookmark na raiz
% e adiciona espaço de parte no Sumário
% ----------------------------------------------------------
\phantompart

% ---
% Conclusão
% ---
\chapter{Conclusão}
% ---

Capítulo conclusão

\lipsum[31-33]

%-------------------------------------------------------------------------------------
%		CONTEÚDO PÓS-TEXTUAL
%		Parte do documento que aparece após o capítulo de conclusão, incluindo 
%		bibliografia e anexos.
%-------------------------------------------------------------------------------------
\postextual

\bibliography{references}

% Consulte o manual da classe abntex2 para orientações sobre o glossário.
%
%\glossary

\begin{apendicesenv}

\partapendices

% ----------------------------------------------------------
\chapter{Quisque libero justo}
% ----------------------------------------------------------

\lipsum[50]

% ----------------------------------------------------------
\chapter{Nullam elementum urna vel imperdiet sodales elit ipsum pharetra ligula
ac pretium ante justo a nulla curabitur tristique arcu eu metus}
% ----------------------------------------------------------
\lipsum[55-57]

\end{apendicesenv}
% ---


% ----------------------------------------------------------
% Anexos
% ----------------------------------------------------------

% ---
% Inicia os anexos
% ---
\begin{anexosenv}

% Imprime uma página indicando o início dos anexos
\partanexos

% ---
\chapter{Morbi ultrices rutrum lorem.}
% ---
\lipsum[30]

% ---
\chapter{Cras non urna sed feugiat cum sociis natoque penatibus et magnis dis
parturient montes nascetur ridiculus mus}
% ---

\lipsum[31]

% ---
\chapter{Fusce facilisis lacinia dui}
% ---

\lipsum[32]

\end{anexosenv}

%---------------------------------------------------------------------
% INDICE REMISSIVO
%---------------------------------------------------------------------
\phantompart
\printindex
%---------------------------------------------------------------------

\end{document}
